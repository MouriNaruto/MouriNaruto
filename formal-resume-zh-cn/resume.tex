% !TEX program = xelatex
% This is my resume
% Chinese translation
% by Kenji Mouri

\documentclass{resume}

\usepackage{lastpage}
\usepackage{fancyhdr}
\usepackage{linespacing_fix} % disable extra space before next section
\usepackage[fallback]{xeCJK}

\linespread{1.11} 

\begin{document}
\renewcommand\headrulewidth{0pt}

\name{卢起}

\basicInfo{
  \email{Mouri\_Naruto@Outlook.com} \textperiodcentered\
  \phone{182-6175-7560} \textperiodcentered\
  \github[MouriNaruto]{https://github.com/MouriNaruto} \textperiodcentered\
}

\section{教育经历}

\datedsubsection{\textbf{常熟理工学院}}{2016.09 -- 现在}
  汽车服务工程专业,已通过 CET-6,预计 2020 年 7 月毕业。

\section{技能}
\begin{itemize}

  \item \textbf{程序语言}:
    熟悉 C/C++、C\#、Python。

  \item \textbf{开发工具}:
    日常在 Windows 下使用 Visual Studio、IDA,同时也经常使用 Git、VSTS 等团队协作工具。

  \item \textbf{Windows 应用开发}:
    有使用 Windows SDK (Win32、COM、ATL、WTL、WRL、C++/CX、C++/WinRT) 开发实际项目的经验。

  \item \textbf{.Net Framework 和 .Net Core 应用开发}:
    有使用 WPF、Windows Forms、ASP.NET Core 开发实际项目的经验。
  
  \item \textbf{Linux 应用开发}:
    有开发 ROS (Kinetic、Melodic) 节点、Qt 图形应用程序的经验。

\end{itemize}

\section{个人项目}

\datedsubsection{\textbf{NSudo}}{\url{https://github.com/M2Team/NSudo}}
一套系统管理工具集,NSudo Launcher 用于在 Windows 下以特定权限和设置运行应用,NSudo Devil Mode 是提供给开发者用的在管理员权限的进程下绕过 Windows 的文件和注册表的访问检查。

\datedsubsection{\textbf{Nagisa}}{\url{https://github.com/Project-Nagisa/Nagisa}}
一个基于 UWP 平台的文件传输工具,支持后台传输,还处于早期开发阶段。

\datedsubsection{\textbf{Mile}}{\url{https://github.com/MouriNaruto/Mile}}
在自己的项目使用的公共库,使用 C++17 的编译器语法特性,但提供 C ABI 兼容。

\datedsubsection{\textbf{MINT}}{\url{https://github.com/Chuyu-Team/MINT}}
自己基于 Process Hacker 的 PHNT 项目制作的 Windows 未文档化的用户模式 API 定义,且对 MSVC 工具链进行了优化。

\datedsubsection{\textbf{libkcrt}}{\url{https://github.com/Chuyu-Team/libkcrt}}
一个正在开发中的在现代 MSVC 工具链下使用 ntdll.dll 中 C 运行时函数的解决方案,未来会和 VC-LTL 进行深度合作。

\section{团队经历}

\datedsubsection{\textbf{初雨团队}}{2014.07 -- 现在}
负责 Dism++ NCleaner 清理增强组件的开发,协助开发 Dism++ 映像热备份功能和多语言适配。实现了 \href{https://github.com/Chuyu-Team/vc-ltl-samples}{VC-LTL 项目的使用示例},且协助开发 VC-LTL 的 Visual Studio 工具集成。

\datedsubsection{\textbf{常熟理工学院车联网大数据实验室 (301实验室)}}{2018.03 -- 现在}
与实验室的其他人一同参与了第十六届“挑战杯”,并获取江苏省大学生课外学术科技作品竞赛二等奖。协助开发实验室物联网平台,增加对西门子 S7-1200 PLC 的支持,且重构和扩充了部分实现。主导了实验室的自动驾驶框架的开发,其中的 Web 服务器是我的毕业设计项目。

\section{其他}
\begin{itemize}

  \item 致力于编写使用最少的语法特性和第三方库的紧凑实现。

  \item 开源贡献:推进和改善了 OpenSSL \href{https://github.com/openssl/openssl/pulls?q=is:pr+author:MouriNaruto+}{官方分支}和\href{https://github.com/microsoft/openssl/pulls?q=is:pr+author:MouriNaruto+}{微软分支}对 UWP 平台的支持、添加了\href{https://github.com/microsoft/winfile/pulls?q=is:pr+author:MouriNaruto+}{微软开源的旧版文件管理器 winfile} 的简体中文支持并共同确定了多语言支持目录结构、改善了 \href{https://github.com/covscript/covscript/pulls?q=is:pr+author:MouriNaruto+}{Covariant Script 编程语言}的 Windows 平台体验、创建了 \href{https://github.com/ffmpeginteropx/FFmpegInteropX/pulls?q=is:pr+author:MouriNaruto+}{FFmpegInteropX} 项目的 \href{https://github.com/ffmpeginteropx/FFmpegInteropX/tree/FFmpegUniversal}{FFmpeg 单动态链接库文件分支}。
  
  \item 技术文章:\href{https://bbs.pediy.com/thread-257345.htm}{《NSudo 恶魔模式 - 一个面向希望无视文件和注册表访问检查的开发者的解决方案》}、\href{https://www.52pojie.cn/thread-512713-1-1.html}{《实现每显示器高 DPI 识别 (Per\-Monitor DPI Aware) 的注意事项》}、\href{https://www.52pojie.cn/thread-506556-1-1.html}{《开启 Win10 的文件资源管理器的每显示器 DPI 缩放 (Per\-Monitor DPI Aware) 支持》}、\href{http://bbs.pcbeta.com/viewthread-1567726-1-1.html}{《浅谈 Windows 10 Build 9879 的磁盘清理的 System Compression》}、\href{http://bbs.pcbeta.com/viewthread-1611980-1-1.html}{《浅谈 Metro App 的沙盒 AppContainer》}、\href{http://bbs.pcbeta.com/viewthread-1524688-1-1.html}{《自定义开始屏幕的大小》}、\href{http://bbs.pcbeta.com/viewthread-1535789-1-1.html}{《反汇编 Windows 系统还原代码的成果》}、\href{http://bbs.pcbeta.com/viewthread-1507617-1-1.html}{《Windows 系统还原新探 (Windows 系统还原的较深入研究)》}。
  
  \item 很喜欢结识各种友人,结识了\href{https://github.com/mingkuang-Chuyu}{冥王}、\href{https://github.com/ice1000/}{ice1000}、\href{https://github.com/mikecovlee}{mikecovlee} 等友人。

\end{itemize}

\end{document}
