% !TEX program = xelatex
% This is my resume
% English translation
% by Kenji Mouri

\documentclass{resume}

\usepackage{lastpage}
\usepackage{fancyhdr}
\usepackage{linespacing_fix} % disable extra space before next section

\setmainfont{FreeSerif}
\setsansfont{FreeSans}
\setmonofont{FreeMono}

\linespread{1.11} 

\begin{document}
\renewcommand\headrulewidth{0pt}

\name{Kenji Mouri}

\basicInfo{
  \email{Kenji.Mouri@outlook.com} \textperiodcentered\
  \homepage{https://mouri.moe}
  \github[MouriNaruto]{https://github.com/MouriNaruto}
}

\section{Introduction}
\begin{itemize}
  \item Hi, I'm Kenji Mouri. But my legal name is Qi Lu which is used only when offline. MouriNaruto, Mouri\_Naruto and Mouri are my typical usernames.
  \item I'm interested in writing the most compact implementations by using the least syntaxes and third-party libraries. Also, I am desired to make more friends which can talk something about technology.
  \item I have created and maintained several open-source projects mostly written in C/C++ at night since 2014.
  \item I'm also a proud Microsoft MVP with Developer Technologies and Windows Development award categories.
  \item I hope my next job can do something about system or bare-metal related software development and maintenance.
\end{itemize}

\section{Professional Experience}

\datedline{\href{https://live.qq.com}{\textbf{Qi'e TV, Tencent}} (C++ Software Development Engineer)}{Dec 2020 -- Present}
\begin{itemize}
  \item Lead the development of the QieLive 2.x Series from scratch to release, adapt for Windows 7 Service Pack 1 or later
  \item Building customized FFmpeg, MSBuild and Qt6 toolchains for QieLive
\end{itemize}

\section{Open Source Experience}

\datedline{\href{https://github.com/M2Team}{\textbf{M2-Team}} (Owner)}{Jun 2015 -- Present}
\begin{itemize}
  \item \href{https://github.com/M2Team/NanaZip}{NanaZip} - File archiver intended for the modern Windows experience
  \item \href{https://github.com/M2Team/NanaBox}{NanaBox} - The third-party lightweight out-of-box-experience oriented Hyper-V client
  \item \href{https://github.com/M2Team/NanaGet}{NanaGet} - Lightweight file transfer utility based on aria2 and XAML Islands
  \item \href{https://github.com/M2Team/NanaRun}{NanaRun} - Application runtime environment customization utility for Windows
\end{itemize} 

\datedline{\href{https://github.com/ProjectMile}{\textbf{Project Mile}} (Owner)}{Nov 2020 -- Present}
\begin{itemize}
  \item \href{https://github.com/ProjectMile/Mile.Xaml}{Mile.Xaml} - Lightweight XAML Islands toolchain with modern Windows controls styles
  \item \href{https://github.com/ProjectMile/Mile.HyperV}{Mile.HyperV} - The lightweight library for Hyper-V guest interfaces
  \item \href{https://github.com/ProjectMile/Mile.Cirno}{Mile.Cirno} - Work in progress 9p client for Windows based on Dokany
  \item \href{https://github.com/ProjectMile/Mile.Uefi}{Mile.Uefi} - UEFI Application SDK for Visual Studio
  \item \href{https://github.com/ProjectMile/Mile.Aria2}{Mile.Aria2} - Customized version of aria2 specialize for MSVC toolchain
\end{itemize} 

\datedline{\href{https://github.com/lvgl}{\textbf{LVGL}} (Maintainer)}{Jan 2021 -- Present}
\begin{itemize}
  \item Maintain implementation of \href{https://github.com/lvgl/lv_port_pc_visual_studio}{LVGL port for Visual Studio (lv\_port\_pc\_visual\_studio)}.
  \item Maintain the implementation of \href{https://github.com/lvgl/lv_drivers/pull/117}{New native Windows driver (win32drv)} and \href{https://github.com/lvgl/lvgl/pull/2701}{Windows file system driver (lv\_fs\_win32)}.
\end{itemize}

\section{Honors}

\datedline{\href{https://mvp.microsoft.com/en-us/PublicProfile/5004706?fullName=Kenji Mouri}{\textbf{Microsoft MVP}} (Developer Technologies, Windows Development)}{Feb 2022 -- Present}

\section{Education}

\datedline{\textbf{Changshu Institute of Technology}, China}{Sep 2016 -- Jul 2020}
  Bachelor of Engineering, Automotive Service Engineering. I also have passed the CET-6 at this period.

\section{Skills}
\begin{itemize}
  \item I'm not limited to any specific programming language, but I usually use C/C++ and C\#. I can adapt to any tools under Windows and Linux, but I prefer Visual Studio under Windows.
  \item I have focused on Windows Development with Win32 API, COM (ATL and WTL) and WinRT (C++/WinRT) since 2014, which I started to participate in the Chuyu Team's next generation (at that time) third-party Windows servicing tool which is called DISM++ today.
  \item I love doing some bare-metal development with Visual Studio under Windows. I had made a Hyper-V guest SDK and a UEFI application SDK to make me happy and achieve my goals.
  \item I also had the experience for ROS (not ROS2) because I was a member of autonomous driving team when I was a college student.
\end{itemize}

\end{document}
