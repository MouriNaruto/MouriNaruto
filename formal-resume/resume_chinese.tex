% !TEX program = xelatex
% This is my resume
% Chinese translation
% by Kenji Mouri

\documentclass{resume}

\usepackage{lastpage}
\usepackage{fancyhdr}
\usepackage{linespacing_fix} % disable extra space before next section
\usepackage[fallback]{xeCJK}

\setmainfont{Noto Serif CJK SC}
\setsansfont{Noto Sans CJK SC}
\setmonofont{Noto Sans Mono CJK SC}

\linespread{1.11} 

\begin{document}
\renewcommand\headrulewidth{0pt}

\name{毛利 (卢起)}

\basicInfo{
  \email{Kenji.Mouri@outlook.com} \textperiodcentered\
  \homepage{https://mouri.moe}
  \github[MouriNaruto]{https://github.com/MouriNaruto} 
}

\section{教育经历}

\datedline{\textbf{常熟理工学院},中国}{2016.09 -- 2020.07}
  工学学士,汽车服务工程专业。

\section{荣誉}

\datedline{\href{https://mvp.microsoft.com/en-us/PublicProfile/5004706?fullName=Kenji Mouri}{\textbf{微软最有价值专家 (MVP)}} (Developer Technologies, Windows Development)}{2022.02 -- 现在}

\section{工作经历}

\datedline{\href{https://live.qq.com}{\textbf{企鹅体育}} (C++ 开发工程师)}{2020.12 -- 现在}
\begin{itemize}
  \item 领导企鹅直播伴侣 2.x 系列从零到一的开发工作,且适配 Windows 7 Service Pack 1。
  \item 为企鹅直播伴侣定制 FFmpeg、MSBuild 和 Qt6 工具链。
\end{itemize}

\section{开源贡献}

\datedline{\href{https://github.com/M2Team}{\textbf{M2-Team}} (所有者)}{2015.06 -- 现在}
\begin{itemize}
  \item \href{https://github.com/M2Team/NanaZip}{NanaZip} - 致力于现代 Windows 体验的文件归档工具
  \item \href{https://github.com/M2Team/NanaBox}{NanaBox} - 致力于开箱即用的轻量级第三方 Hyper-V 虚拟化客户端
  \item \href{https://github.com/M2Team/NanaGet}{NanaGet} - 基于 aria2 和 XAML Islands 的轻量级文件传输工具
  \item \href{https://github.com/M2Team/NanaRun}{NanaRun} - Windows 应用运行时环境定制工具
\end{itemize} 

\datedline{\href{https://github.com/ProjectMile}{\textbf{Project Mile}} (所有者)}{2020.11 -- 现在}
\begin{itemize}
  \item \href{https://github.com/ProjectMile/Mile.Xaml}{Mile.Xaml} - 提供现代 Windows 控件风格的轻量级 XAML Islands 工具链
  \item \href{https://github.com/ProjectMile/Mile.Uefi}{Mile.Uefi} - Visual Studio 用 UEFI 应用程序 SDK
  \item \href{https://github.com/ProjectMile/Mile.FFmpeg}{Mile.FFmpeg} - 单文件版 FFmpeg 动态链接库
  \item \href{https://github.com/ProjectMile/Mile.Aria2}{Mile.Aria2} - MSVC 工具链定制版 aria2
\end{itemize} 

\datedline{\href{https://github.com/Chuyu-Team}{\textbf{Chuyu Team}} (成员)}{2014.07 -- 现在}
\begin{itemize}
  \item 协助开发 \href{https://github.com/Chuyu-Team/Dism-Multi-language/releases/latest}{DISM++} 项目,引入 NCleaner 清理增强组件、映像热备份和多语言支持。
  \item 独立开发 \href{https://github.com/Chuyu-Team/MINT}{MINT} 项目,基于 PHNT 项目制作的 Windows 未文档化的用户模式 API 定义。
  \item 独立开发 \href{https://github.com/Chuyu-Team/libkcrt}{libkcrt} 项目,现代 MSVC 工具链下连接到 ntdll.dll 中 CRT 函数的解决方案。
  \item 协助开发 \href{https://github.com/Chuyu-Team/VC-LTL5}{VC-LTL} 项目,引入 VC-LTL 的 Rust 语言支持。
\end{itemize}

\datedline{\href{https://github.com/lvgl}{\textbf{LVGL}} (贡献者)}{2021.01 -- 现在}
\begin{itemize}
  \item 重构并增强\href{https://github.com/lvgl/lv_port_pc_visual_studio}{LVGL 的 Visual Studio 移植版本 (lv\_port\_pc\_visual\_studio)} 的实现。
  \item 引入\href{https://github.com/lvgl/lv_drivers/pull/117}{新式 Windows 原生驱动 (win32drv)} and \href{https://github.com/lvgl/lvgl/pull/2701}{Windows 文件系统驱动 (lv\_fs\_win32)} 的实现。
\end{itemize}

\section{愿景}
\begin{itemize}

  \item 继续重构和增强 NanaZip 并完善 NanaBox 和 NanaGet。
  
  \item 通过将使用 VMBus 传输协议的增强会话模式移植到 Linux 来宾来增强 Hyper-V 和 NanaBox 的用户体验。
  
  \item 实现称为 RaySoul 的具有 Linux 系统调用支持的 64 位 RISC-V 用户空间虚拟机。
  
\end{itemize}

\section{技能}
\begin{itemize}

  \item \textbf{编程语言}:
    \textbf{基本不受特定语言限制},日常使用 C/C++ 和 C\#。

  \item \textbf{Windows 平台开发}:
    \textbf{9 年}开发经验,日常使用 Win32 API、COM (ATL and WTL) 和 WinRT (C++/WinRT)。

  \item \textbf{.NET 应用开发}:
    \textbf{4 年}开发经验,有使用 WPF、WinForms、ASP.NET Core、Xamarin 开发实际项目的经验。
  
  \item \textbf{Linux 应用开发}:
    \textbf{4 年}开发经验,有使用 ROS (Kinetic、Melodic) 节点和 Qt 开发实际项目的经验。

  \item \textbf{开发工具}:
    \textbf{能适应 Windows 和 Linux 下的开发工具},平常在 Windows 下使用 Visual Studio。

\end{itemize}

\section{其他}
\begin{itemize}

  \item 语言:English - 良好(已通过 CET-6),汉语 - 母语。

  \item 致力于编写使用最少的语法特性和第三方库的紧凑实现。

  \item \href{https://github.com/search?q=is%3Apr%20author%3AMouriNaruto&type=pullrequests}{开源贡献}:向 \href{https://gitlab.freedesktop.org/mesa/mesa/-/merge_requests/22961}{Mesa3D}、ncnn、OpenSSL、Windows File Manager 等项目贡献过代码。
  
  \item 在 My Digital Life Forums、远景论坛、吾爱破解论坛、看雪安全论坛以 Mouri\_Naruto 发表过精华或置顶帖。
  
  \item 渴望结识更多的可以进行技术交流的友人。
  
\end{itemize}

\end{document}
